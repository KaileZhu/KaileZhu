 %==============================
\section{\textcolor{blue}{Preliminaries}}\label{sec:preliminary}
%==============================
\subsection{Linear Temporal Logic (LTL)}\label{subsec:LTL}
Linear Temporal Logic (LTL) formulas are composed of a set of atomic propositions $AP$
in addition to several Boolean and temporal operators. Atomic propositions are Boolean variables that can
be either true or false.  Particularly, the syntax~\citep{baier2008principles} is given as follows:
$\varphi \triangleq \top \;|\; p  \;|\; \varphi_1 \wedge \varphi_2  \;|\; \neg \varphi  \;|\; \bigcirc \varphi  \;|\;  \varphi_1 \,\textsf{U}\, \varphi_2,$
where $\top\triangleq \texttt{True}$, $p \in AP$, $\bigcirc$ (\emph{next}),
$\textsf{U}$ (\emph{until}) and $\bot\triangleq \neg \top$.
For brevity, we omit the derivations of other operators like $\Box$ (\emph{always}),
 $\Diamond$ (\emph{eventually}), $\Rightarrow$ (\emph{implication}).
The full semantics and syntax of LTL are omitted here for brevity,
see e.g.,~\citep{baier2008principles}.

An infinite {word} $w$ over the alphabet $2^{AP}$ is defined as an
infinite sequence ${w}=\sigma_1\sigma_2\cdots, \sigma_i\in 2^{AP}$.
The language of $\varphi$ is defined as the set of words that satisfy $\varphi$,
namely, $L_\varphi=Words(\varphi)=\{\boldsymbol{w}\,|\,w\models\varphi\}$
and $\models$ is the satisfaction relation.
However, there is a special class of LTL formula called \emph{syntactically co-safe} formulas \textcolor{blue}{(sc-LTL)} mentioned \textcolor{blue}{in the Definition~2.3 of \citep{belta2017formal}},
which can be satisfied by a set of finite sequence of words.
They only contain the temporal operators $\bigcirc$, $\textsf{U}$ and $\Diamond$
 and are written in positive normal form \textcolor{blue}{where the negation
   operator $\neg$ is not allowed before the temporal
   operators including~$\bigcirc, \textsf{U}, \Diamond$.}

%==============================
\subsection{Nondeterministic B\"uchi Automaton}\label{subsec:nba}

Given an LTL formula $\varphi$ mentioned above, the associated Nondeterministic B\"{u}chi Automaton (NBA)~\citep{baier2008principles} can be derived with the following structure.
%====================
\begin{definition}[NBA] \label{def:nba}
A NBA $\mathcal{B}$ is a 5-tuple: $\mathcal{B}=(Q,\,Q_0,\,\Sigma,\,\delta,\,\\Q_F)$,
where $Q$ is the set of states;
$Q_0\subseteq Q$ is the set of initial states;
$\Sigma=AP$ is the allowed alphabet;
$\delta:Q\times \Sigma\rightarrow2^{Q}$ is the transition relation;
$Q_F\subseteq Q$ is the set of \emph{accepting} states. \hfill $\blacksquare$
\end{definition}
%====================

Given an infinite word $w=\sigma_1\sigma_2\cdots$, \textcolor{blue}{we can get a set of resulting \emph{runs} within $\mathcal{B}$.}
Each of them is an infinite sequence $\rho=q_0q_1q_2\cdots$
such that $q_0\in Q_0$, and $q_i\in Q$, $q_{i+1}\in\delta(q_i,\,\sigma_i)$ hold for all index~$i\geq 0$.
%A run is called \emph{accepting} if it holds that
%$\inf(\rho)\bigcap {Q}_F \neq \emptyset$,
%where $\inf(\rho)$ is the set of states that appear in $\rho$ infinitely often.
%In general, an accepting run can be written in the prefix-suffix structure~\citep{baier2008principles},
%where the prefix starts from an initial state and ends in an accepting state;
%and the suffix is a cyclic path that contains the same accepting state.
%It is easy to show that such a run is accepting if the prefix is experienced once,
%while the suffix is repeated infinitely often.
%In general, the size of~$\mathcal{B}$ is double exponential to the size of~$|\varphi|$.
The set of all words that correspond to \emph{accepting} runs is called
the \emph{language} generated by the NBA. The language can be expressed
in the prefix-suffix structure~\textcolor{blue}{\citep{baier2008principles}},
\textcolor{blue}{while the prefix part
can result in a run from an initial state to an accepting state, and the suffix
part can result in a cyclic run that contains the same accepting state.}
\textcolor{blue}{In addition, the
satisfaction of a \textcolor{blue}{sc-LTL} formula can be achieved in a finite time, i.e.,
each word satisfying an \textcolor{blue}{sc-LTL} formula consists of a satisfying prefix
and an arbitrary suffix.}
%In general, the size of~$\mathcal{B}$ is double exponential to the size of~$|\varphi|$.

%==============================
\subsection{Partially Ordered Set}\label{subsec:partial}
A partial order~\citep{simovici2008mathematical} defined on a set $S$ is a relation $\rho\subseteq S\times S$,
which is reflexive, antisymmetric, and transitive.
Then, the pair $(S,\, \rho)$ is referred to as a partially ordered set (or simply \emph{poset}).
A generic partial order relation is given by $\leq$.
Namely, for $x,\,y\in S$, $(x,\,y)\in \rho$ if $x\leq y$.
The set $S$ is totally ordered if $\forall x,\,y\in S$, either $x\leq y$ or $y\leq x$ holds.
Two elements $x,\,y$ are incomparable if neither $x\leq y$ nor $y\leq x$ holds, denoted by $x\parallel y$.
