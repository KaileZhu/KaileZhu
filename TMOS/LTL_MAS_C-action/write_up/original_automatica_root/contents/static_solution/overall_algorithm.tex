\subsection{Overall Algorithm}\label{subsubsec:overall-algorithm}

%=======================================
\begin{algorithm}[t]
	\caption{Complete algorithm for time minimization
        under collaborative temporal tasks}
	\label{alg:complete}
	\SetKwInOut{Input}{Input}
        \SetKwInOut{Output}{Output}
	\Input {Task formula~$\varphi$, time budget~$t_0$.}
	\Output {Assignment~$J^\star$,  makespan~$T^\star$}
	Compute $\mathcal{B}_{\varphi}$ given~$\varphi$ \\
	Compute $\mathcal{B}^{-}_{\varphi}$ by pruning~$\mathcal{B}_{\varphi}$
        \tcp*{Sec.~\ref{subsubsec:NBA-pruning}}
        Initialize $\mathcal{J}=\emptyset$;\\
        \While{$time<t_0$}{
	$P_{\varphi}\leftarrow \texttt{compute\_poset}(\mathcal{B}^{-}_{\varphi})$
        \label{algline:comp-poset} \tcp*{Alg.~\ref{alg:compute-poset}}
        $(J,\, T, \,T_\Omega)\leftarrow \texttt{BnB}(P_{\varphi})$
        \label{algline:BnB} \tcp*{Alg.~\ref{alg:BnB}}
        Store $(J,\,T,\,T_\Omega)$ in $\mathcal{J}$;\\
        }
        Select~$J^\star$, $T^\star_\Omega$ with minimum $T^\star$ among $\mathcal{J}$;\\
        Get time list $T_\Omega$ for each task;\\
	\textbf{Return} $J^\star$, $T^\star$, $T^\star_\Omega$;
\end{algorithm}
%=======================================

The complete algorithm can be obtained by combining
Alg.~\ref{alg:compute-poset} to compute posets
and Alg.~\ref{alg:BnB} to assign sub tasks in the posets.
More specifically, as summarized in Alg.~\ref{alg:complete},
the NBA associated with the given task~$\varphi$ is derived and pruned as
described in Sec.~\ref{alg:compute-poset}.
Afterwards, within the allowed time budget~$t_0$,
once the set of posets~$\mathcal{P}_{\varphi}$ derived from
Alg.~\ref{alg:compute-poset}
is nonempty, any poset $P_\varphi\in \mathcal{P}_{\varphi}$ is fed to
the task assignment Alg.~\ref{alg:BnB} to compute the current best
assignment~$J$ and its makespan~$T$, which is stored in
a solution set~$\mathcal{J}$.
This procedure is repeated until the computation time elapsed.
By then, the optimal assignment~$J^\star$ and its makespan~$T^\star$
are returned as the optimal solution.

\begin{remark}\label{remark:parallel}
It is worth noting that even though Alg.~\ref{alg:compute-poset} and
Alg.~\ref{alg:BnB} are presented sequentially in
Line~\ref{algline:comp-poset}~-~\ref{algline:BnB}.
They can be implemented and run {in parallel}, i.e., more posets are
generated and stored in Line~\ref{algline:comp-poset},
while other posets are used for task assignment in Line~\ref{algline:BnB}.
Moreover, it should be emphasized that Alg.~\ref{alg:complete} is an \emph{anytime}
algorithm meaning that it can run for any given time budget and generate the
current best solution.
As more time is allowed, either better solutions are found or confirmations are
given that no better solutions exist.
%% As discussed in Sec.~\ref{sec:introduction}, this property is particularly useful
%% for real-time applications where computation time is limited.
\hfill  $\blacksquare$
\end{remark}




Finally, once the optimal plan~$J^\star$ is computed with the format
defined in~\eqref{eq:node}, i.e.,
the action sequence $\tau_n$ and is assigned
to agent~$n$ with the associated time stamps~$t_{\omega_1}t_{\omega_2}\cdots t_{\omega_n}$ in $T^\star_\Omega$.
In other words,
agent~$n$ can simply execute this sequence of subtasks at the designated time,
namely $\omega_k$ at time~$T_{\omega_k}$.
Then, it is ensured that all task can be fulfilled in minimum time.
However, such way of execution can be prone to uncertainties in system model
such as fluctuations in action duration
and failures during execution, which will be discussed in the next section.
