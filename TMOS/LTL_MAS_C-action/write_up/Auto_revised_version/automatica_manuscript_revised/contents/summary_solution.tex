%==============================
\section{Algorithmic Summary}\label{subsec:summary}

To summarize, the proposed planning algorithm in Alg.~\ref{alg:complete}
can be used offline to synthesize the complete plan that
minimizes the time for accomplishing the specified collaborative temporal tasks.
During execution, the proposed online synchronization scheme can be applied to
overcome uncertainties in the duration of certain transition or actions.
Moreover, whenever one or several agents have failures, the proposed adaptation
algorithm can be followed to re-assign the remaining unfinished tasks.
In the rest of this section,
we present the analysis of completeness, optimality and computational
complexity for Alg.~\ref{alg:complete}.

\begin{theorem}[Completeness]\label{theo:completeness}
Given enough time, Alg.~\ref{alg:complete} can return the optimal
assignment~$J^\star$ with minimum makespan~$T^\star$.
\end{theorem}
\begin{proof}
 To begin with, Lemma~\ref{lemma:accepting-poset} shows that any R-poset obtained
 by Alg.~\ref{alg:compute-poset} is accepting.
 As proven in Lemma~\ref{lemma:complete-poset},
 the underlying DFS search scheme finds all accepting runs of~$\mathcal{B}^-_{\varphi}$ via exhaustive search.
 Thus, the complete R-posets~$\mathcal{P}_{\varphi}$ returned by
 Alg.~\ref{alg:compute-poset} after termination is ensured to cover all
 accepting words of the original task.
 Moreover, Lemma~\ref{lemma:BnB-satisfying} shows that any assignment~$J^\star$
 from the BnB Alg.~\ref{alg:BnB} satisfies the input R-poset
 from Alg.~\ref{alg:compute-poset}.
 Combining these two lemmas, it follows that any assignment~$J^\star$
 from Alg.~\ref{alg:complete} satisfies the original task formula.
 Second, since both Algs.~\ref{alg:compute-poset} and~\ref{alg:BnB} are exhaustive,
 the complete set of R-posets and the complete search tree of all
 possible assignments under each R-poset are searched for the optimal solution.
 As a result, once both sets are enumerated, the derived assignment~$J^\star$ is
 optimal over all possible solutions.
\end{proof}

The computational complexity of Alg.~\ref{alg:complete}
is analyzed as follows. To generate one valid R-poset in Alg.~\ref{alg:compute-poset},
the worst case time complexity is~$\mathcal{O}(M^2)$, where~$M$ is the maximum
number of subtasks within the given task thus bounded by the number of edges in the
pruned NBA~$\mathcal{B}^-_{\varphi}$.
However, as mentioned in Sec.~\ref{subsec:nba},
the size of~$\mathcal{B}$ is double exponential to the size of~$|\varphi|$.
The number of R-posets is upper bounded by the number of accepting runs
within~$\mathcal{B}^-_{\varphi}$, thus worst-case combinatorial to
the number of nodes within~$\mathcal{B}^-_{\varphi}$.
Furthermore, regarding the BnB search algorithm, the search space in the worst
case is~$\mathcal{O}(M!\cdot N^M)$ as the possible sequence of all subtasks
is combinatorial and the possible assignment is exponential to the
number of agents.
However, the worst-time complexity to compute the upper bound via Alg.~\ref{alg:BnB}
remains~$\mathcal{O}(M\cdot N)$ as it greedily assigns the remaining subtasks,
while the complexity to compute the lower bound is~$\mathcal{O}(M^2)$ as it relies on a BFS over
the R-poset graph~$\mathcal{G}_{P_\varphi}$.
How to decompose the overall formula while ensuring
the satisfaction of each subformula, thus overcoming the bottleneck
in the size of~$\mathcal{B}_{\varphi}$, remains our ongoing work.

%------------------------------
\begin{remark}\label{remark:anytime}
The exponential complexity above is expected due to the NP-hardness of the considered problem.
However, as emphasized earlier, the main contribution of the proposed algorithm
is the anytime property.
Namely, it can return the best solution within the given time budget,
which is particularly useful for real-time applications where computation time is limited.
\hfill  $\blacksquare$
\end{remark}
%------------------------------
