%==============================
\section{Introduction}\label{sec:introduction}

% Overview of MAS coordination
\IEEEPARstart{M}{ulti}-agent systems consist of a fleet of homogeneous or heterogeneous robots,
such as autonomous ground vehicles and aerial vehicles.
They are often deployed to accomplish tasks that are otherwise too inefficient or even infeasible for a single robot~\cite{arai2002advances}.
First,
by allowing the robots to move and act concurrently,
the overall efficiency of the team can be significantly improved.
For instance,
a fleet of delivery vehicles can drastically reduce the delivery time~\cite{toth2002overview};
and a team of drones can surveil a large terrain and detect poachers~\cite{cliff2015online}.
Second,
by enabling multiple robots to directly collaborate on a task,
capabilities of the team can be greatly extended.
For instance,
several mobile manipulators can transport objects that are otherwise too heavy for one~\cite{fink2008multi};
and a team of mobile vehicles can collaboratively herd moving targets via formation~\cite{varava2017herding}.
% Temporal tasks as LTL formulas
Furthermore, to specify these complex tasks,
many recent work propose to use formal languages such as Linear Temporal Logic (LTL)
formulas~\cite{baier2008principles}, as an intuitive yet powerful way to describe
both spatial and temporal requirements on the team behavior,
see~\cite{ulusoy2013optimality, kantaros2020stylus, schillinger2018simultaneous, guo2018multirobot} for examples.


%========================================
\begin{table*}[t]
  \begin{center}
    \caption{Comparison of related work as discussed in Sec.~\ref{subsec:multi-ltl},
  regarding the problem formulation and key features.}
\label{table:compare}
%--------------------
\begin{adjustbox}{width=0.8\linewidth}
{\def\arraystretch{1.2}\tabcolsep=3pt
\begin{tabular}{cccccccc}
\toprule
Ref. & Syntax & Collaboration & Objective  & Solution & Anytime & Synchronization & Adaptation \\ \midrule
\cite{guo2015multi, tumova2016multi} & Local-LTL & No & Summed Cost & Dijkstra & No & Event-based  & Yes\\
\cite{guo2016task} & Local-LTL & Yes & Summed Cost & Dijkstra &  No & Event-based  & Yes \\
\cite{kantaros2020stylus, luo2021abstraction} & Global-LTL & No & Summed Cost & Sampling & Yes & All-time & No \\
\cite{schillinger2018simultaneous} & Global-LTL & No & Balanced & Martins' Alg.  & No & None & No \\
\cite{luo2021temporal} & Global-LTL & Yes & Summed Cost & MILP & No & All-time & No \\
\cite{sahin2019multirobot, jones2019scratchs} & Global-cLTL & No & Summed Cost & MILP & No & Partial & No \\
\textbf{Ours} & Global-LTL & Yes & Min Time & BnB & Yes & Event-based & Yes \\
\bottomrule
\end{tabular}%
}
\end{adjustbox}
%--------------------
\end{center}
\end{table*}
%========================================

% Time minimization, computational complexity, anytime algorithm
However, coordination of these robots to accomplish the desired task can result in great complexity,
as the set of possible task assignments can be combinatorial with respect to the number of robots
and the length of tasks~\cite{toth2002overview}.
It is particularly so when certain metric should be minimized,
such as the completion time or the summed cost of all robots.
Considerable results are obtained in many recent work, e.g.,
via optimal planning algorithms
such as mixed integer linear programming (MILP) in~\cite{luo2021temporal,
 sahin2019multirobot, jones2019scratchs},
and search algorithms over state or solution space as
in~\cite{kantaros2020stylus, schillinger2018simultaneous, luo2021abstraction}.
Whereas being sound and optimal,
many existing solutions are designed from an offline perspective,
thus lacking in one of the following aspects that could be essential for robotic missions:
(I) \emph{Real-time requirements}.
Optimal solutions often take an unpredictable amount of time to compute,
without any intermediate feedback.
However, for many practical applications,
good solutions that are generated fast and reliably are often more beneficial;
(II) \emph{Synchronization} during plan execution.
Many of the derived plans are assigned to the robots
and executed independently without any synchronization among them,
e.g., no coordination on the start and finish time of a subtask.
Such procedure generally relies on a precise model of the underlying system
such as traveling time and task execution time,
and the assumption that no direct collaborations are required.
Thus, any mismatch in the given model or dependency in the subtasks would lead to erroneous execution;
(III) \emph{Online adaptation}.
An optimal but static solution can not handle changes in the workspace or in the team during
online execution, e.g., certain paths between subtasks are blocked, or some robots break down.


% paper overview
To overcome these challenges,
this work takes into account the minimum-time task coordination problem of a team of heterogeneous robots.
The team-wise task is given as LTL formulas over the desired actions to perform at the
desired regions of interest in the environment.
Such action can be independent that it can be done by one of these robots alone,
or collaborative where several robots should collaborate to accomplish it.
The objective is to find an optimal task policy for the team such that the completion time for the task is minimized.
Due to the NP-completeness of this problem,
the focus here is to design an anytime algorithm that returns the best solution within the given time budget.
More specifically,
the proposed algorithm interleaves between the partial ordering analysis
of the underlying task automaton for task decomposition,
and the branch and bound (BnB) search method for task assignment.
Each of these two sub-routines is anytime itself.
The proposed partial relations can be applied to non-instantaneous subtasks,
thus providing a more general model for analyzing concurrent subtasks.
Furthermore, the proposed lower and upper bounding methods during the BnB search
significantly reduces the search space.
The overall algorithm is proven to be complete and sound for the considered objective,
and also shown empirically that a feasible and near-optimal solution is quickly reached.
Besides,
an online synchronization protocol is proposed to handle fluctuations in the execution time,
while ensuring that the partial ordering constraints are still respected.
Lastly, to handle agent failures during the task execution,
an adaptation algorithm is proposed to dynamically reassign unfinished subtasks
online to maintain optimality.
The effectiveness and advantages of the proposed algorithm
are demonstrated rigorously via numerical simulations and hardware experiments,
particularly over {large-scale} systems of more than $10$ robots and $30$ subtasks.


%contribution
The main contribution of this work is threefold:
(I) it extends the existing work on temporal task planning to allow collaborative subtasks
and the practical objective of minimizing task completion time;
(II) it proposes an anytime algorithm that is sound, complete and optimal,
which is particularly suitable for real-time applications where computation time is restricted;
and (III) it provides a novel theoretical approach that combines the partial ordering
analysis for task decomposition and the BnB search for task assignment.

%structure
The rest of the paper is organized as follows:
Sec.~\ref{sec:related-work} reviews related work.
The formal problem description is given in Sec.~\ref{sec:problem}.
Main components of the proposed framework are presented in
Sec.~\ref{subsec:initial-synthesis}-\ref{subsec:summary}.
Experiment studies are shown in Sec.~\ref{sec:experiments},
followed by conclusions and future work in Sec.~\ref{sec:conclusion}.
