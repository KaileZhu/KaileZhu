\subsection{Task Assignment}\label{subsubsec:task-assignment}
Given the set of posets~$\mathcal{P}_{\varphi}$ derived from the previous
section, this section describes how this set can be used to compute
the optimal assignment of these subtasks. More specifically, we consider
the following sub-problems of task assignment:

\begin{problem}\label{problem:}
Given any poset $P=(\Omega,\, \preceq_{\varphi},\, \neq_{\varphi})$
where~$P\in \mathcal{P}_{\varphi}$,
find the optimal assignment of all subtasks in~$\Omega$ to the multi-agent system
$\mathcal{N}$ such that
(i) all partial ordering requirements in $\preceq_{\varphi},\, \neq_{\varphi}$ are
respected; (ii) the maximum completion time of all subtasks is minimized.
\hfill $\blacksquare$
\end{problem}



To begin with, the above problem is an NP-hard as it includes the multi-vehicle routing
problem~\cite{gini2017multi, khamis2015multi},
and the job-shop scheduling problem~\cite{brucker1994branch} as special instances which are
 well-known to be NP-hard. Here we proposed an anytime assignment algorithm
based on the Branch and Bound (BnB) search
method~\cite{lawler1966branch, morrison2016branch}.
It is not only complete and optimal, but also anytime, meaning that a good
solution can be inquired within any given time budget.
As shown in Fig.~\ref{fig:bnb_search_logic},
the four typical components of a BnB
algorithm are the node expansion, the branching method,
and the design of the upper and lower bounds.
These components for our application are described in detail below. 


 

%========================================
\begin{figure}[t!]
	\centering
	\includegraphics[width=0.8\linewidth]{figures/bnb_graph3.pdf}
	%--------------------
	\caption{
		Illustration of the main components in the BnB search,
		i.e., the node expansion and branching to generate explore new nodes (in green arrow);
		and the lower and upper bounding to avoid undesired branches (in orange).}
	\label{fig:bnb_search_logic}
	%--------------------
\end{figure}
%========================================



\textbf{Node expansion}.
Each node in the search tree stands for one partial assignment of the
subtasks, i.e.,
\begin{equation}\label{eq:node}
\nu = (\tau_1,\,\tau_2,\cdots,\tau_N),
\end{equation}
where $\tau_n$ is the ordered sequence of tasks assigned to agent~$n\in \mathcal{N}$.
To avoid producing infeasible nodes, we assign only the next task satisfies
the partial order. Let~$\nu$ be the current node, $\Omega_\nu$ be the assigned
task and $\Omega^-_\nu$ be the unassigned task. The \emph{next} subtask~$\omega'$ 
to be assigned is chosen from the poset~$\mathcal{G}_{P_\varphi}$ defined in
 Def.~\ref{def:poset-graph}:
\begin{equation}\label{eq:next-task}
\omega' \in \Omega_\nu, \; \forall \omega' \in \text{Pre}(\omega),
\end{equation}
where~$\text{Pre}(\omega)$ is the set of preceding or parenting subtasks of $\omega$ in~$\mathcal{G}_{P_\varphi}$. 

\textbf{Branching}.
Given the set of nodes to be expanded,
the branching method determines the order in which these child nodes are
visited.
We propose to use $A^\star$ search here as the heuristic function matches well
with the lower bounds introduced in the sequel.
 

%==============================
\begin{algorithm}[t]\footnotesize
\caption{$\texttt{upper\_bound}(\cdot)$: Compute the upper bound of solutions
rooted from a node}
\label{alg:upper_bound}
\SetKwInOut{Input}{Input}\SetKwInOut{Output}{Output}
\Input {Poset~$P_{\varphi}$, node~$\nu$.}
\Output {Assignment~$\overline{J}_\nu$, upper bound~$\overline{T}_\nu$.}
\While(\tcp*[f]{\eqref{eq:node-tasks}}){$\Omega^-_\nu \neq \emptyset$}{
\ForAll{$\omega\in \Omega^-_\nu$\, \text{and}\, $\{n\}\subset\mathcal{N} $ satisfy $\omega$
\label{algline:all-next}}
{Compute and save $\nu_{n,\omega}$ by assigning $\omega$ to agent set~$\{n\}$ }
Select~$\nu^{+}_{\star}=\textbf{argmax}_{\nu\in \{\nu_{n,\omega}^+\}}\,\{\eta_{\nu}\}$
\label{algline:compute-eta}\tcp*{\eqref{eq:node-makespan}}
$\nu \leftarrow \nu^{+}_{\star}$; \label{algline:expand-node}\\
}
Compute and makespan~$T_\nu$ and subtask start time $T_\Omega$ of assignment~$J_\nu$ \\
\label{algline:return}
\textbf{Return} $\overline{J}_\nu=J_\nu,\, \overline{T}_\nu=T_\nu$,$\overline{T}_\Omega=T_\Omega$;\\
\end{algorithm}
%==============================

\textbf{Lower and upper bounding}.
The lower bound method is designed to check whether a node fetched by \emph{branching}
has the potential to produce a better solution.
And the upper bound method is tried for each chosen node to update
 the current best solution.
More specifically, given a node~$\nu$,
the upper bound of all solutions rooted from this node
is estimated via a greedy task assignment policy,
as summarized in Alg.~\ref{alg:upper_bound}:
\begin{equation}\label{eq:upper-bound}
\overline{J}_\nu,\, \overline{T}_\nu ,\overline{T}_\Omega= \texttt{upper\_bound}(\nu,\, P_{\varphi}),
\end{equation}
{where~$\overline{T}_\nu$ is upper bound, and~$\overline{J}_\nu$ is the
associated complete assignment with the same structure of $\nu$, while its $\Omega^-$ is empty; $\overline{T}_\Omega$ is the beginning time
of each subtasks.}
From node~$\nu$, any task~$\omega\in \Omega^-_\nu$ is assigned
to any allowed agent set~$\{n\}\subset\mathcal{N}$ in Line~\ref{algline:all-next},
thus generating a set of child nodes~$\{\nu^+_{n,\omega}\}$.
Then, for each node~$\nu\in \{\nu^+_{n,\omega}\}$,
its \emph{concurrency} level~$\eta_{\nu}$ is estimated as follows:
\begin{equation}\label{eq:node-makespan}
T_\nu = \max_{n\in\mathcal{N}} \{T_{\tau_n}\},\;
T^{\texttt{s}}_\nu = \sum_{\omega\in\Omega_\nu}\, D_{\omega}N_\omega,\;
\eta_\nu = \frac{T^{\texttt{s}}_\nu}{T_\nu},
\end{equation}
where node~$\nu=(\tau_1,\cdots,\tau_N)$;~$T_{\tau_n}$ is the execution
time of all subtasks in~$\tau_n$ by agent~$n$;
$T_\nu$ is the max current makespan calculate by \eqref{eq:node-makespan};
and $T^{\texttt{s}}_\nu$ is the total execution time of all subtasks~$\omega$
given its duration~$D_\omega$ and the number of participants~$N_\omega$.
Thus, the child node with the highest~$\eta_{\nu}$ is chosen as
the next node to expand in Line~\ref{algline:compute-eta}-\ref{algline:expand-node}.
This procedure is repeated until no subtasks remain unassigned.
Afterwards, once a complete assignment~$J_\nu$ is generated, its makespan $T_v$, and start time $T_\Omega$ of subtasks is obtained by
solving a simple linear program as in Line~\ref{algline:return}.

Furthermore, the lower bound of the makespan of all solutions rooted from this
node is estimated via two separate relaxations of the original problem:
one is to consider only the partial ordering constraints while ignoring
the agent capacities;
another is vice versa. The details of optimization functions for the upper(lower) bound 
can be found in the supplementary material.


%==============================
\begin{algorithm}[t]\footnotesize
\caption{$\texttt{BnB}(\cdot)$: Anytime BnB algorithm for task assignment}
\label{alg:BnB}
\SetKwInOut{Input}{Input}\SetKwInOut{Output}{Output}
\Input {Agents $\mathcal{N}$, poset $P_{\varphi}$, time budget $t_0$.}
\Output {Best assignment $J^\star$ and makespan $T^\star$.}
Initialize root node~$\nu_0$ and queue $Q=\{(\nu_0,0)\}$; \label{algline:init-start}\\
Set $T^\star=\infty$ and $J^\star=()$;\label{algline:init-end}\\
\While{($Q$ not empty) and ($time<t_0$)}{
Take node $\nu$ off $Q$;\\
$\overline{J}_\nu,\, \overline{T}_\nu,\overline{T}_\Omega  = \texttt{upper\_bound}(\nu,\, P_{\varphi})$
\tcp*{Alg.~\ref{alg:upper_bound}} \label{algline:upper}

\If{$T^\star > \overline{T}_\nu$ \label{algline:update1}}{
Set $T^\star=\overline{T}_\nu$ and $J^\star=\overline{J}_\nu$, $T^\star_\Omega=\overline{T}_\Omega$;\label{algline:update2}}
Expand child nodes $\{\nu^+\}$ from $\nu$ ;\\
\ForAll{$\nu^+\in\{\nu^+\}$}{
	$\underline{T}_\nu = \texttt{lower\_bound}(\nu^+,\,P_{\varphi})$
 \label{algline:lower}\\
	\If{$\underline{T}_\nu \leq T^\star$ \label{algline:store1}}{
			Add $(\nu^+,\underline{T}_\nu)$ into $Q$.\label{algline:store2}
	}
	}
}
\textbf{Return} $J^\star, T^\star, T^\star_\Omega$;
\end{algorithm}
%==============================



%==============================

%------------------------------
 
\todoliu{
Given the above components, the complete BnB algorithm can be stated as
in Alg.~\ref{alg:BnB}.
In the initialization step in
Line~\ref{algline:init-start}-\ref{algline:init-end},
the root node~$\nu_0$ is created as an empty assignment,
the estimated optimal cost~$T^{\star}$ is set to infinity,
and the queue to store un-visited nodes $Q$ and the lower bound as indexes contains only $(\nu_0,0)$.
Then, within the time budget, a node $\nu$ is taken from $Q$ for expansion with smallest lower bound.
We calculate the upper bound of $\nu$ in line~\ref{algline:upper} and update the optimal value $T^\star,J^\star,T^\star_\Omega$
in line~\ref{algline:update1}-\ref{algline:update2}.
After that, we expand the child nodes $\{\nu^+\}$ of current node $\nu$.
Finally, we calculate the lower bound of each new node $\nu^+$ in line~\ref{algline:lower} and only store
the node with potential to get a better solution into $Q$ in line~\ref{algline:store1},\ref{algline:store2}.
This process repeat itself until time elapsed or the whole search tree is
exhausted.
}
